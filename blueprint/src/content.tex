% Turán project — blueprint content
% This file is included by both web.tex and print.tex. Do NOT add a \begin{document} here.

% Turán project — blueprint content
% This file is included by both web.tex and print.tex. Do NOT add a \begin{document} here.

%%%%%%%%%%%%%%%%%%%%%%%%%%%%%%%%%%%%%%%%%%%%%%%%%%%%%%%%%%%%%%%%%%%%%%%%%%%%%%%%
% Preliminaries
%%%%%%%%%%%%%%%%%%%%%%%%%%%%%%%%%%%%%%%%%%%%%%%%%%%%%%%%%%%%%%%%%%%%%%%%%%%%%%%%

%%%%%%%%%%%%%%%%%%%%%%%%%%%%%%%%%%%%%%%%%%%%%%%%%%%%%%%%%%%%%%%%%%%%%%%%%%%%%%%%
% Section 1 — Concentrating support on a clique
%%%%%%%%%%%%%%%%%%%%%%%%%%%%%%%%%%%%%%%%%%%%%%%%%%%%%%%%%%%%%%%%%%%%%%%%%%%%%%%%
\section{Concentrating support on a clique - Improve Operartion}

\begin{definition}[A better distribution]
    \label{def:Better}
    \lean{Better}
Given a weight function \(W\), define \(w^* := \texttt{Better}(W)\) such that \(\mathrm{supp}(w^*) \subseteq \mathrm{supp}(W)\) and 
\[
W.\mathrm{fw} \leq w^*.\mathrm{fw}.
\]
In words: \(w^*\) is a distribution improving or preserving the edge weight and supported within that of \(W\).
\end{definition}

\begin{definition}[Single transfer]
    \label{def:Improve}
    \lean{Improve}
Given distinct vertices \(v_j \neq v_i\), define \(w' := \texttt{Improve}(W, v_j, v_i)\) by transferring a small amount of weight from \(v_j\) (loose) to \(v_i\) (gain):
\[
w'(v_j) = W(v_j) - \varepsilon, \quad w'(v_i) = W(v_i) + \varepsilon,
\]
for some small \(\varepsilon > 0\), and \(w'(v) = W(v)\) for \(v \neq v_i, v_j\).
In words: weight is shifted from \(v_j\) to \(v_i\) preserving total weight.
\end{definition}

\begin{lemma}[Sum splitting along the partition]
    \label{lem:Improve_partition_sum_split}
    \lean{Improve_partition_sum_split}
The sum over edges \(E\) of the function \(\texttt{vp}\) splits as
\[
\sum_{e \in E} \texttt{vp}(w', e) = \sum_{\substack{e \in E \\ e \ni v_i}} \texttt{vp}(w', e) + \sum_{\substack{e \in E \\ e \ni v_j}} \texttt{vp}(w', e) + \sum_{\substack{e \in E \\ e \not\ni v_i, v_j}} \texttt{vp}(w', e).
\]
In words: the total sum decomposes into parts incident to \(v_i\), incident to \(v_j\), and the rest.
\end{lemma}

\begin{lemma}[Gain-incidence increases]
    \label{lem:Improve_gain_contribution_increase}
    \lean{Improve_gain_contribution_increase}
The increment in the sum over edges incident to \(v_i\) is
\[
\sum_{e \ni v_i} \texttt{vp}(w', e) - \sum_{e \ni v_i} \texttt{vp}(W, e) = \varepsilon \sum_{v_k \in N(v_i)} W(v_k),
\]
where \(N(v_i)\) is the neighborhood of \(v_i\).
In words: the gain vertex's contribution increases by \(\varepsilon\) times the sum of weights of its neighbors.
\end{lemma}

\begin{lemma}[Loose-incidence becomes zero]
    \label{lem:Improve_loose_contribution_zero}
    \lean{Improve_loose_contribution_zero}
The sum over edges incident to \(v_j\) after the transfer satisfies
\[
\sum_{e \ni v_j} \texttt{vp}(w', e) = 0.
\]
In words: the loose vertex's incident edge contributions vanish after the transfer.
\end{lemma}

\begin{lemma}[Unchanged complement]
    \label{lem:Improve_unchanged_edge_sum}
    \lean{Improve_unchanged_edge_sum}
For edges \(e\) not incident to \(v_i\) or \(v_j\),
\[
\texttt{vp}(w', e) = \texttt{vp}(W, e).
\]
In words: edges outside the gain and loose neighborhoods remain unchanged.
\end{lemma}

\begin{lemma}[Transfer does not decrease \(\mathrm{fw}\)]
    \label{lem:Improve_total_weight_nondec}
    \lean{Improve_total_weight_nondec}
    \uses{lem:Improve_partition_sum_split, lem:Improve_unchanged_edge_sum, lem:Improve_gain_contribution_increase, lem:Improve_loose_contribution_zero}
Under the neighborhood condition (i.e., appropriate adjacency relations), we have
\[
W.\mathrm{fw} \leq w'.\mathrm{fw}.
\]
In words: the total edge weight does not decrease after applying \texttt{Improve}.
\end{lemma}

\begin{lemma}[Improve strictly reduces support]
    \label{lem:Improve_support_strictly_reduced}
    \lean{Improve_support_strictly_reduced}
    \uses{def:Improve}
If the neighborhood condition fails, then
\[
|\mathrm{supp}(w')| < |\mathrm{supp}(W)|.
\]
In words: the transfer strictly reduces the number of vertices with positive weight.
\end{lemma}

\begin{theorem}[Support of \texttt{Better} is a clique]
    \label{thm:Better_forms_clique}
    \lean{Better_forms_clique}
    \uses{def:Improve, lem:Improve_total_weight_nondec, lem:Improve_support_strictly_reduced}
The support of \(w^* := \texttt{Better}(W)\) forms a clique:
\[
\forall v_a, v_b \in \mathrm{supp}(w^*), v_a \neq v_b \implies \{v_a, v_b\} \in E.
\]
In words: every two distinct vertices with positive weight in \(w^*\) are adjacent.
\end{theorem}


%%%%%%%%%%%%%%%%%%%%%%%%%%%%%%%%%%%%%%%%%%%%%%%%%%%%%%%%%%%%%%%%%%%%%%%%%%%%%%%%
% Section 2 — Two–vertex enhance step
%%%%%%%%%%%%%%%%%%%%%%%%%%%%%%%%%%%%%%%%%%%%%%%%%%%%%%%%%%%%%%%%%%%%%%%%%%%%%%%%
\section{The Enhance Operation}

\begin{definition}[Enhance]
  \label{def:Enhance}
  \lean{Enhance}
  \leanok
Given distinct non-adjacent vertices \(v_j, v_i\), define \(w' := \texttt{Enhance}(W, v_j, v_i, \varepsilon)\) by transferring \(\varepsilon > 0\) weight from \(v_j\) to \(v_i\):
\[
w'(v_j) = W(v_j) - \varepsilon, \quad w'(v_i) = W(v_i) + \varepsilon, \quad w'(v) = W(v) \text{ for } v \neq v_i, v_j,
\]
with the condition \(\{v_j, v_i\} \notin E\).
In words: Enhance transfers weight between non-adjacent vertices to increase edge weight.
\end{definition}

\begin{lemma}[Sum over support]
  \label{lem:sum_over_support}
  \lean{sum_over_support}
  \leanok
The total vertex weight satisfies
\[
\sum_{v \in \mathrm{supp}(W)} W(v) = 1.
\]
In words: the weights sum to 1 over the support.
\end{lemma}

\begin{lemma}[Supported edge partition]
  \label{lem:supported_edge_partition}
  \lean{supported_edge_partition}
  \leanok
The edge set \(E\) partitions as
\[
E = E_{v_i} \cup E_{v_j} \cup E_{\mathrm{rest}},
\]
where
\[
E_{v_i} = \{e \in E : v_i \in e\}, \quad E_{v_j} = \{e \in E : v_j \in e\}, \quad E_{\mathrm{rest}} = E \setminus (E_{v_i} \cup E_{v_j}).
\]
In words: edges are split into those incident to \(v_i\), to \(v_j\), and the rest.
\end{lemma}

\begin{lemma}[Enhance gain sum]
  \label{lem:Enhance_gain_sum}
  \lean{Enhance_gain_sum}
  \leanok
Under \autoref{def:Enhance}, the change in the sum over edges incident to \(v_i\) satisfies
\[
\sum_{e \in E_{v_i}} \texttt{vp}(w', e) - \sum_{e \in E_{v_i}} \texttt{vp}(W, e) = \varepsilon \sum_{v_k \in N(v_i)} W(v_k).
\]
In words: the gain vertex's edge contribution increases by \(\varepsilon\) times the sum of its neighbors' weights.
\end{lemma}

\begin{lemma}[Enhance loose sum]
  \label{lem:Enhance_loose_sum}
  \lean{Enhance_loose_sum}
  \leanok
Under \autoref{def:Enhance}, the sum over edges incident to \(v_j\) satisfies
\[
\sum_{e \in E_{v_j}} \texttt{vp}(w', e) = 0.
\]
In words: the loose vertex's incident edge contributions become zero after Enhance.
\end{lemma}

\begin{definition}[Bijection inside the clique]
    \label{def:the_bij}
    \lean{the_bij}
    \leanok
Define a bijection
\[
\phi: \{ e \in E_{v_j} \setminus \{s(v_j, v_i)\} \} \to \{ e \in E_{v_i} \setminus \{s(v_j, v_i)\} \}
\]
mapping edges incident to \(v_j\) (except \(s(v_j,v_i)\)) to edges incident to \(v_i\) (except \(s(v_j,v_i)\)).
In words: this bijection pairs edges incident to \(v_j\) with edges incident to \(v_i\) within the clique.
\end{definition}

\begin{lemma}[Bijection preserves]
  \label{lem:the_bij_same}
  \lean{the_bij_same}
  \leanok
For any edge \(e\) incident to \(v_j\) (excluding \(s(v_j, v_i)\)), the "other" vertex weight satisfies
\[
W(\mathrm{other}(e, v_j)) = W(\mathrm{other}(\phi(e), v_i)).
\]
In words: the bijection preserves weights at the other endpoints of edges.
\end{lemma}

\begin{lemma}[Loose/gain equality]
  \label{lem:Enhance_sum_loose_gain_equal}
  \lean{Enhance_sum_loose_gain_equal}
  \leanok
  \uses{def:the_bij, lem:the_bij_same}
The total weight transfer balances edge contributions:
\[
\sum_{e \in E_{v_j}} \texttt{vp}(w', e) + \sum_{e \in E_{v_i}} \texttt{vp}(w', e) \geq \sum_{e \in E_{v_j}} \texttt{vp}(W, e) + \sum_{e \in E_{v_i}} \texttt{vp}(W, e).
\]
In words: the combined edge contributions of loose and gain vertices do not decrease after Enhance.
\end{lemma}

\begin{lemma}[Complement unchanged]
  \label{lem:Enhance_sum_complement_unchanged}
  \lean{Enhance_sum_complement_unchanged}
  \leanok
For edges \(e \in E_{\mathrm{rest}}\),
\[
\texttt{vp}(w', e) = \texttt{vp}(W, e).
\]
In words: edges not incident to \(v_i\) or \(v_j\) remain unaffected by Enhance.
\end{lemma}

\begin{lemma}[Edge contribution increase]
  \label{lem:Enhance_edge_gainloose_increase}
  \lean{Enhance_edge_gainloose_increase}
  \leanok
The net edge contribution satisfies
\[
\sum_{e \in E_{v_i} \cup E_{v_j}} \texttt{vp}(w', e) \geq \sum_{e \in E_{v_i} \cup E_{v_j}} \texttt{vp}(W, e).
\]
In words: the total contribution from gain and loose vertices does not decrease.
\end{lemma}

\begin{lemma}[Support edges unchanged]
  \label{lem:Enhance_support_edges_same}
  \lean{Enhance_support_edges_same}
  \leanok
For any vertex \(v \notin \{v_i, v_j\}\), the edge contributions satisfy
\[
\sum_{e \ni v} \texttt{vp}(w', e) = \sum_{e \ni v} \texttt{vp}(W, e).
\]
In words: vertices outside gain and loose retain their edge contributions after Enhance.
\end{lemma}

\begin{theorem}[Enhance increases edge weight]
  \label{thm:Enhance_total_weight_stricinc}
  \lean{Enhance_total_weight_stricinc}
  \leanok
  \uses{lem:supported_edge_partition,
        lem:Enhance_gain_sum,
        lem:Enhance_loose_sum,
        lem:Enhance_sum_loose_gain_equal,
        lem:Enhance_sum_complement_unchanged,
        lem:Enhance_edge_gainloose_increase,
        lem:Enhance_support_edges_same}
Applying \texttt{Enhance} strictly increases the total edge weight:
\[
w'.\mathrm{fw} > W.\mathrm{fw}.
\]
In words: the Enhance operation strictly improves the total edge weight.
\end{theorem}

%%%%%%%%%%%%%%%%%%%%%%%%%%%%%%%%%%%%%%%%%%%%%%%%%%%%%%%%%%%%%%%%%%%%%%%%%%%%%%%%
% Section 3 — Equalising weights on the clique
%%%%%%%%%%%%%%%%%%%%%%%%%%%%%%%%%%%%%%%%%%%%%%%%%%%%%%%%%%%%%%%%%%%%%%%%%%%%%%%%
\section{Equalizing the weights on the clique - EnhanceD}

\begin{definition}[Carefully chosen \(\varepsilon\)]
  \label{def:the_eps}
  \lean{the_eps}
  \leanok
Define
\[
\mathsf{the\_\varepsilon} := \max \left\{ \varepsilon > 0 \mid w'(v_j) = W(v_j) - \varepsilon \geq 0, \quad w'(v_i) = W(v_i) + \varepsilon \leq \frac{1}{|\mathrm{supp}(W)|} \right\}.
\]
In words: \(\mathsf{the\_\varepsilon}\) is the maximal \(\varepsilon\) transferring weight from the argmax vertex \(v_j\) to the argmin vertex \(v_i\) without violating support constraints.
\end{definition}

\begin{definition}[Maximising the number of uniform vertices]
  \label{def:max_uniform_support}
  \lean{max_uniform_support}
  \leanok
Define
\[
m := \max \{ k \mid \exists w \text{ with } \mathrm{supp}(w) \subseteq \mathrm{supp}(W),\ w.\mathrm{fw} \geq W.\mathrm{fw},\ \text{and } w(v) = \frac{1}{k} \text{ for at least } k \text{ vertices} \}.
\]
In words: \(m\) is the maximal number of vertices with uniform weight \(1/k\) achievable without decreasing edge weight.
\end{definition}

\begin{lemma}[Best uniform distribution exists]
  \label{lem:exists_best_uniform}
  \lean{exists_best_uniform}
  \uses{def:max_uniform_support}
  \leanok
There exists \(w_M\) with \(\mathrm{supp}(w_M) \subseteq \mathrm{supp}(W)\), \(w_M.\mathrm{fw} \geq W.\mathrm{fw}\), and with at least \(m\) vertices having weight \(1/m\).
In words: a maximiser \(w_M\) achieving the maximal uniform vertex count exists.
\end{lemma}

\begin{definition}[UniformBetter]
  \label{def:UniformBetter}
  \lean{UniformBetter}
  \uses{lem:exists_best_uniform}
  \leanok
Define \(w_M := \texttt{UniformBetter}(W)\) as such a maximiser with maximal uniform support.
\end{definition}

\begin{definition}[Enhanced]
  \label{def:Enhanced}
  \lean{Enhanced}
  \uses{def:Enhance, def:the_eps}
  \leanok
Define
\[
w^+ := \texttt{Enhanced}(W) := \texttt{Enhance}(W, v_j, v_i, \mathsf{the\_\varepsilon}),
\]
where \(v_j = \arg\max_{v} W(v)\), \(v_i = \arg\min_{v} W(v)\).
In words: Enhanced transfers maximal weight from the heaviest to the lightest vertex.
\end{definition}

\begin{lemma}
    \label{lem:Enhanced_unaffected}
    \lean{Enhanced_unaffected}
    \uses{def:Enhance, def:Enhanced}
For any vertex \(v\) with \(W(v) = \frac{1}{|\mathrm{supp}(W)|}\),
\[
w^+(v) = W(v).
\]
In words: vertices already at uniform weight remain unchanged under Enhanced.
\end{lemma}

\begin{lemma}
    \label{lem:Enhanced_effect_argmax}
    \lean{Enhanced_effect_argmax}
    \uses{def:Enhance, def:Enhanced}
The weight at the argmax vertex \(v_j\) after Enhanced satisfies
\[
w^+(v_j) = \frac{1}{|\mathrm{supp}(W)|}.
\]
In words: Enhanced reduces the argmax vertex's weight to the uniform level.
\end{lemma}

\begin{lemma}
    \label{lem:Enhanced_inc_uniform_count}
    \lean{Enhanced_inc_uniform_count}
    \uses{def:Enhanced, lem:Enhanced_effect_argmax, lem:Enhanced_unaffected}
The number of vertices with weight \(\frac{1}{|\mathrm{supp}(W)|}\) increases after Enhanced:
\[
|\{ v : w^+(v) = 1/|\mathrm{supp}(W)| \}| > |\{ v : W(v) = 1/|\mathrm{supp}(W)| \}|.
\]
In words: Enhanced increases the count of uniform weight vertices.
\end{lemma}

\begin{lemma}
    \label{lem:UniformBetter_support_equiv}
    \uses{def:UniformBetter}
    \leanok
The support of \(W\) forms a clique if and only if the support of \(w_M = \texttt{UniformBetter}(W)\) forms a clique:
\[
\mathrm{supp}(W) \text{ clique } \iff \mathrm{supp}(w_M) \text{ clique}.
\]
In words: the clique property is preserved by UniformBetter.
\end{lemma}

\begin{lemma}[Uniform weights on the support]
  \label{lem:UniformBetter_constant_support}
  \lean{UniformBetter_constant_support}
  \uses{def:UniformBetter, def:Enhanced, lem:UniformBetter_support_equiv, lem:Enhanced_inc_uniform_count, thm:Enhance_total_weight_stricinc}
  \leanok
For every vertex \(v \in \mathrm{supp}(w_M)\),
\[
w_M(v) = \frac{1}{|\mathrm{supp}(w_M)|}.
\]
In words: the weights of all support vertices in UniformBetter are uniform.
\end{lemma}

\begin{lemma}[Edge values under UniformBetter]
  \label{lem:UniformBetter_edges_value}
  \lean{UniformBetter_edges_value}
  \uses{lem:UniformBetter_constant_support}
  \leanok
For any edge \(e = \{v_a, v_b\}\) with \(v_a, v_b \in \mathrm{supp}(w_M)\),
\[
\texttt{vp}(w_M, e) = \left(\frac{1}{|\mathrm{supp}(w_M)|}\right)^2.
\]
In words: every supported edge has value equal to the square of the uniform vertex weight.
\end{lemma}

\begin{lemma}[Edge count in a clique]
  \label{lem:clique_size}
  \lean{clique_size}
  \uses{lem:UniformBetter_facts}
  \leanok
If \(|\mathrm{supp}(w_M)| = k\), then
\[
|\{ e \in E : e \subseteq \mathrm{supp}(w_M) \}| = \frac{k(k-1)}{2}.
\]
In words: the supported edges form a complete graph on \(k\) vertices.
\end{lemma}

\begin{lemma}[A light computation]
  \label{lem:computation}
  \lean{computation}
  \leanok
For \(k > 0\),
\[
\frac{k(k-1)}{2} \cdot \left(\frac{1}{k}\right)^2 = \frac{1}{2}\left(1 - \frac{1}{k}\right).
\]
In words: the total edge weight for a clique with uniform weights simplifies to \(\frac{1}{2}(1 - \frac{1}{k})\).
\end{lemma}

\begin{lemma}[Monotonicity of the bound]
  \label{lem:bound_real}
  \lean{bound}
  \lean{bound_real}
  \leanok
The function
\[
f(k) := \frac{1}{2}\left(1 - \frac{1}{k}\right)
\]
is nondecreasing for \(k \geq 1\).
In words: the bound increases as the clique size increases.
\end{lemma}

\begin{theorem}[Final bound inside a clique]
  \label{lem:finale_bound}
  \lean{finale_bound}
  \uses{lem:Better_forms_clique, lem:Better_non_decr, lem:Better_forms_clique, lem:UniformBetter_fw_ge, def:UniformBetter, lem:UniformBetter_edges_value, lem:UniformBetter_clique, lem:bound_real, lem:computation, lem:clique_size, lem:UniformBetter_edges_value}
  \leanok
If \(W\) is supported on a clique of size \(k \leq p-1\), then
\[
W.\mathrm{fw} \leq \frac{1}{2}\left(1 - \frac{1}{p-1}\right).
\]
In words: the total edge weight is bounded by the Turán bound for cliques of size less than \(p\).
\end{theorem}

\begin{definition}[Uniform weights over all vertices]
  \label{def:UnivFun}
  \lean{UnivFun}
  \leanok
Define
\[
\texttt{UnivFun}(G)(v) := \frac{1}{|V|} \quad \forall v \in V.
\]
In words: the uniform vertex weight function assigns equal weight \(1/|V|\) to every vertex.
\end{definition}

\begin{lemma}[Total weight under \(\texttt{UnivFun}\)]
  \label{lem:UnivFun_weight}
  \lean{UnivFun_weight}
  \uses{def:UnivFun}
  \leanok
The total edge weight satisfies
\[
(\texttt{UnivFun}(G)).\mathrm{fw} = |E| \cdot \left(\frac{1}{|V|}\right)^2.
\]
In words: the total edge weight under uniform vertex weights equals the number of edges times the square of the uniform weight.
\end{lemma}

\begin{theorem}[Turán's Theorem]
  \label{thm:turans}
  \lean{turans}
  \uses{def:UnivFun, lem:UnivFun_weight, lem:finale_bound, lem:computation}
  \leanok
Let \(p \geq 2\) and let \(G\) be a \(p\)-clique-free graph. Then
\[
|E| \leq \frac{1}{2}\left(1 - \frac{1}{p-1}\right) |V|^2.
\]
In words: the number of edges in a \(p\)-clique-free graph is bounded by Turán's theorem.
\end{theorem}